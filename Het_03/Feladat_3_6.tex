\documentclass{article}
\usepackage[magyar,english]{babel}
\usepackage{t1enc}
\frenchspacing
\usepackage{listings}
\usepackage[svgnames]{xcolor}
\usepackage{float}
\newfloat{python_program}{hbt!}{lop}
\lstdefinestyle{python_stilus}{language=python,%
tabsize=2,%
showspaces=false,%
showtabs=false,%
numbers=left,%
stepnumber=4,%
frame=lRtB,%
framexleftmargin=20pt}
\newfloat{c_program}{hbt!}{lop}
\lstdefinestyle{c_stilus}{language=c,%
backgroundcolor=\color{gray!20},%
keywordstyle=\color{DarkRed},%
identifierstyle=\color{DarkBlue},%
commentstyle=\color{gray},%
tabsize=2,%
showspaces=false,%
showtabs=false,%
numbers=left,%
stepnumber=4,%
framexleftmargin=20pt}
\renewcommand{\lstlistlistingname}{Programkódok jegyzéke}
\renewcommand{\lstlistingname}{Programkód}
\usepackage{hyperref}

\begin{document}
\lstlistoflistings
\clearpage
\begin{python_program}
\lstinputlisting[caption={Bináris keresés és beszúró rendezés Python-ban},style=python_stilus]{Programok/bin_ker.py}
\end{python_program}
\begin{c_program}
\lstinputlisting[caption={Bináris keresés C-ben},style=c_stilus,firstline=13]{Programok/binsearch.c}
\end{c_program}
\end{document}