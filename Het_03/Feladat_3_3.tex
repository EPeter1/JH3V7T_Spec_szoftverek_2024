\documentclass{article}
\usepackage[magyar]{babel}
\usepackage{t1enc}
\frenchspacing

\begin{document}
Ez egy teljesen szokásos mondat kezdete, most viszont következik egy LaTeX parancs, ami a megadott szöveget \textit{dőlt betűssé} teszi: \verb|\textit{dőlt betűs szöveg}|. Ábrák úsztatására a figure környezet által van lehetőség, melynek módja az ábrával kapcsolatos kód elhelyezése a \verb|\begin{figure}| és \verb|\end{figure}| közötti részekben. Ha feliratozni is szeretnénk az így beszúrt képet, akkor az megtehető a \verb|\caption{képaláírás}| segítségével.\par
Kedvenc gyümölcsök:
\begin{enumerate}
\item szőlő,
\item narancs,
\item áfonya,
\item kiwi.
\end{enumerate}\par
A fenti lista LaTeX-ben a következő módon néz ki:
\begin{verbatim}
Kedvenc gyümölcsök:
\begin{enumerate}
\item szőlő,
\item narancs,
\item áfonya,
\item kiwi.
\end{enumerate}\par
\end{verbatim}
\end{document}