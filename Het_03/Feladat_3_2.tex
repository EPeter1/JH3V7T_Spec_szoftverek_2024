\documentclass{article}
\usepackage[magyar]{babel}
\usepackage{t1enc}
\frenchspacing
\usepackage{array}
\usepackage[table]{xcolor}
\usepackage{subcaption}
\usepackage{multirow}
\usepackage{hyperref}

\begin{document}
\listoftables
\clearpage
\begin{table}
\caption{Igazítás, rácsvonalak}
\centering
\begin{tabular}{p{30pt}||rcl|}
& egy & kettő & három \\ \hline \hline
Helló világ! & négy & öt & hat \\ \cline{2-4}
& hét & nyolc & kilenc \\ \cline{1-1} \cline{3-4}
lórum ipse & tíz & & tizenkettő \\ \hline
\end{tabular}
\end{table}
\begin{table}
\caption{Színezés}
\begin{subtable}[b]{0.5\linewidth}
\caption{Zebra}
\centering
\rowcolors{2}{blue!20}{yellow!20}
\begin{tabular}{r|c|l}
egy & kettő & három \\ \hline
négy & öt & hat \\
hét & nyolc & kilenc \\
tíz & tizenegy & tizenkettő \\
\end{tabular}
\end{subtable}
\begin{subtable}[b]{0.5\linewidth}
\caption{Tarka}
\centering
\newcolumntype{s}{>{\columncolor{gray}\color{white}}r}
\arrayrulecolor{green}
\begin{tabular}{s|cl}
\textcolor{green}{egy} & \multicolumn{1}{c}{kettő} & három \\ \hline
négy & öt & hat \\
hét & nyolc & \cellcolor{green!20}kilenc \\
tíz & tizenegy & tizenkettő \\
\end{tabular}
\end{subtable}
\end{table}
\begin{table}
\caption{Cellák összevonása}
\centering
\arrayrulecolor{black}
\begin{tabular}{r|c|c|}
egy & \multicolumn{2}{c|}{kettő} \\ \hline
\multirow{2}{2em}{négy} & öt & hat \\ \cline{2-3}
& \multicolumn{2}{c|}{\multirow{2}{2em}{nyolc}} \\ \cline{1-1}
tíz & \multicolumn{2}{c|}{} \\ \hline
\end{tabular}
\end{table}
\end{document}