\documentclass[]{article}
\usepackage[english,magyar]{babel}
\usepackage{t1enc}
\frenchspacing

\begin{document}
\texttt{Ez egy typewriter betűtípus.}
\newline
\textbf{Ez a mondat vastag betűvel íródott.}
\newline\newline
\textsc{kiskapitális szó}
\newline
\textit{dőlt betűs szó}
\newline\newline
\begin{sffamily}
\textsl{slanted betűforma Sans Serif-be ágyazva}
\end{sffamily}
\newline\newline
\emph{\textit{dőlt betűs kiemelés}
\newline
\textbf{vastag betűs kiemelés}
\newline
\textsc{kiskapitális kiemelés}}
\newline\newline
Most jön egy nagyobb betűméretű \Large{kifejezés}, a következő szó pedig legyen \MakeUppercase{nagybetűs}!
\end{document}