\documentclass[twoside]{article}
\usepackage[magyar]{babel}
\usepackage{t1enc}
\usepackage{hulipsum}
\usepackage{geometry}
\frenchspacing
\setcounter{tocdepth}{5}
\setcounter{secnumdepth}{5}
\renewcommand{\thefootnote}{\fnsymbol{footnote}}
\geometry{margin=3cm,outer=5cm}
\geometry{bindingoffset=1cm}
\geometry{marginparwidth=3cm,marginparsep=0.5cm}

\begin{document}
\pagenumbering{gobble}
\title{Book osztályú zagyva szöveg}
\maketitle
\clearpage
\begin{abstract}
\hulipsum[1-2]
\footnote{Lábjegyzet}
\end{abstract}
\clearpage
\pagenumbering{roman}
\tableofcontents
\clearpage
~
\clearpage
\pagenumbering{arabic}
\section[1. szakasz]{Rövid cím}
\footnote{Lábjegyzet}
\subsection{Egyik alszakasz}
\marginpar{Széljegyzet}
\hulipsum[3-7]
\subsection{Másik alszakasz}
\hulipsum[8-11]
\section[2. szakasz]{Ez most egy nagyon hosszú, többsoros szakaszcím}
\subsection{Harmadik alszakasz}
\hulipsum[12-14]
\subsubsection{Beágyazott szakasz}
\hulipsum[15-17]
\paragraph{Beágyazottabb szakasz}
\hulipsum[18-20]
\marginpar{Széljegyzet}
\subparagraph{Legbeágyazottabb szakasz}
\hulipsum[21-23]
\appendix
\clearpage
\section{Egyik függelék}
\subsection{Egyik alfüggelék}
\hulipsum[24]
\subsection{Másik alfüggelék}
\hulipsum[25]
\clearpage
\section{Másik függelék}
\subsection{Harmadik alfüggelék}
\subsubsection{Beágyazott alfüggelék}
\hulipsum[26-27]
\subsubsection{Beágyazott alfüggelék}
\hulipsum[28]
\end{document}