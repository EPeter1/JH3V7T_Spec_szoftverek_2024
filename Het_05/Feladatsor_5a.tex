\documentclass[aspectratio=169,12pt,xcolor={table}]{beamer}
\usepackage[magyar]{babel}
\usepackage{t1enc}
\frenchspacing
\usepackage{hulipsum}
\title{5a. feladatsor (LaTeX)}
\author{Erdélyi Péter}
\institute{Miskolci Egyetem}
\date{2024}
\newtheorem{tetel}{Tétel}
\def\sectionname{Szakasz}
\AtBeginSection{\frame{\sectionpage}%
\frame{\tableofcontents[currentsection]}}
\usetheme{Berkeley}
\usecolortheme{dolphin}

\begin{document}
\frame{\titlepage}
\section{1. feladat}
\subsection{Frame-ek}
\begin{frame}{Frame környezet címe}{Alcím}
\transblindshorizontal
A legelső $frame$ környezet diája látható itt.
\end{frame}
\begin{frame}{Egy dummy frame}{Dummy alcím}
\transduration{2}
\begin{itemize}
\item<1,3-4> szöveg eleje
\item<2,4> szöveg közepe
\item<1-2> szöveg vége
\end{itemize}
\end{frame}
\subsection{Verbatim és zagyva szöveg}
\begin{frame}[fragile]{Verbatim használata}
\transblindshorizontal
\begin{verbatim}
Kedvenc gyümölcsök:
\begin{enumerate}
\item szőlő,
\item narancs,
\item áfonya,
\item kiwi.
\end{enumerate}\par
\end{verbatim}
\end{frame}
\begin{frame}[allowframebreaks]{Zagyva szöveg generálása}{Széttördelés több frame-re}
\hulipsum[3]
\end{frame}
\section{2. feladat}
\subsection{Columns és block elemek}
\begin{frame}{Columns környezet használata}
\transwipe<2>[direction=0]
\begin{columns}[c]
\begin{column}{.5\linewidth}
\begin{itemize}
\item felsorolás
\item felsorolás
\end{itemize}
\begin{enumerate}
\item számozott lista
\item számozott lista
\item számozott lista
\end{enumerate}
\end{column}
\begin{column}{.5\linewidth}
\begin{figure}
\includegraphics<1>[width=6cm,height=4cm]{../Het_03/Kepek/Caernarfon_castle_front.png}
\only<1>{\caption{Caernarfoni vár (elölnézet)}}
\includegraphics<2>[width=3.2cm,height=4cm]{../Het_03/Kepek/Caernarfon_castle_court.png}
\only<2>{\caption{Caernarfoni vár (udvar)}}
\end{figure}
\end{column}
\end{columns}
\end{frame}
\begin{frame}{Példák block környezetre}
\begin{block}{Sima $block$ címe}
Sima $block$ tartalma
\end{block}\pause
\begin{exampleblock}{$Exampleblock$ címe}
\begin{itemize}
\item felsorolás
\item felsorolás
\end{itemize}
\end{exampleblock}\pause
\begin{alertblock}{}
\alert{Cím nélküli $alertblock$}
\end{alertblock}
\end{frame}
\subsection{Matematikai környezetek}
\begin{frame}{Matematikai környezetek tesztelése}{Theorem, proof}
\begin{tetel}<1>[XY nagyon fontos tétele]
A tétel állítása kimondja, hogy ...
\end{tetel}
\begin{proof}<2>[XY nagyon fontos tételének bizonyítása]
Most pedig bizonyítjuk a fenti tételt. Tudjuk, hogy ...
\end{proof}
\end{frame}
\subsection{Semiverbatim környezet}
\begin{frame}[fragile]{Semiverbatim használata}
\begin{semiverbatim}
\transboxin
\color{blue}\\begin\{frame\}\alert{\{Semiverbatim környezet\}}
\textcolor{black}{A környezet tartalma}
\\end\{frame\}
\end{semiverbatim}
\end{frame}
\end{document}