\documentclass{article}
\usepackage[magyar]{babel}
\usepackage{t1enc}
\frenchspacing
\usepackage{amsthm}
\newtheorem{tetelek}{Tétel}
\newtheorem{lemmak}[tetelek]{Lemma}
\theoremstyle{definition}
\newtheorem{definiciok}{Definíció}[section]
\usepackage{hyperref}

\begin{document}
\section{Első szakasz}
\begin{definiciok}
Természetes számoknak nevezzük a nemnegatív egész számokat.
\end{definiciok}
\begin{definiciok}
Egy pozitív egész szám prímtényezőin a szám prímszám osztóinak összességét értjük.
\end{definiciok}
\begin{tetelek}
Minden 1-nél nagyobb természetes szám felbontható prímtényezők szorzatára, és ez a felbontás egyértelmű.
\end{tetelek}
\section{Második szakasz}
\begin{tetelek}[Thalész]
Ha vesszük egy $O$ középpontú kör $AB$ átmérőjét, valamint a körvonal egy tetszőleges ($A$-tól és $B$-től különböző) $C$ pontját, akkor az $ABC$ háromszög $C$ csúcsánál lévő $\gamma$ szöge derékszög lesz.
\end{tetelek}
\begin{definiciok}
Egy egész együtthatós polinomot primitívnek nevezünk, ha az együtthatóinak legnagyobb közös osztója 1.
\end{definiciok}
\begin{lemmak}[Gauss]
Ha két polinom primitív, akkor a szorzatuk is primitív.
\end{lemmak}
\begin{proof}[Thalész-tétel bizonyítása]
Azt fogjuk felhasználni, hogy a háromszög belső szögeinek összege 180°. Legyen $O$ a kör középpontja. Ekkor az $AOC$ és $COB$ háromszög egyenlő szárú, azaz ...
\end{proof}
\begin{proof}[Gauss-lemma bizonyítása]
A bizonyítás elérhető \href{https://hu.wikipedia.org/wiki/Gauss-lemma#A_lemma_bizonyítása}{ezen} a linken.
\end{proof}
\end{document}