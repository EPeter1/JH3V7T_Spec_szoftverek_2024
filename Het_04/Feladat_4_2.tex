\documentclass{article}
\usepackage[magyar]{babel}
\usepackage{t1enc}
\frenchspacing
\usepackage{amsmath}
\usepackage{amssymb}
\DeclareMathOperator{\sign}{sgn}
\usepackage{hyperref}

\begin{document}
\section{Bevezető}
Az $\frac{1}{n^2}$ sorösszege:
\[ \sum_{n=1}^\infty \frac{1}{n^2} = \frac{\pi^2}{6} \text{.} \]
Az $n!$ ($n$ faktoriális) a számok szorzata 1-től $n$-ig, azaz
\begin{equation}\label{eq:faktorialis}
n! := \prod_{k = 1}^n k = 1 \cdot 2 \cdot ... \cdot n \text{.}
\end{equation}
Konvenció szerint $0! = 1$.\\
Legyen $0 \leq k \leq n$. A binomiális együttható
\[ \binom{n}{k} := \frac{n!}{k! \cdot (n - k)!} \text{,} \]
ahol a faktoriálist (\ref{eq:faktorialis}) szerint definiáljuk.\\
Az előjel- azaz szignum függvényt a következőképpen definiáljuk:
\[ \sign(x) :=
\begin{cases}
1, & \text{ha } x > 0, \\
0, & \text{ha } x = 0, \\
-1, & \text{ha } x < 0.
\end{cases} \]
\clearpage
\section{Determináns}
Legyen
\[ [n] := \{1, 2, \cdots, n\} \]
a természetes számok halmaza 1-től $n$-ig.\\
Egy $n$-edrendű \textit{permutáció} $\sigma$ egy bijekció $[n]$-ből $[n]$-be. Az $n$-edrendű permutációk halmazát, az ún. szimmetrikus csoportot, $S_n$-el jelöljük.\\
Egy $\sigma \in S_n$ permutációban inverziónak nevezünk egy $(i, j)$ párt, ha $i < j$ de $\sigma_i > \sigma_j$.\\
Egy $\sigma \in S_n$ permutáció paritásának az inverziók számát nevezzük:
\[ \mathcal{I} (\sigma) := \Bigl| \bigl\{ (i, j) \text{ } \big| \text{ } i, j \in [n], i < j, \sigma_i > \sigma_j \bigr\} \Bigr| \text{.} \]
Legyen $A \in \mathbb{R}^{n \times n}$, egy ${n \times n}$-es (négyzetes) valós mátrix:
\[ A = \left( \begin{matrix}
a_{11} & a_{12} & \cdots & a_{1n} \\
a_{21} & a_{22} & \cdots & a_{2n} \\
\vdots & \vdots & \ddots & \vdots \\
a_{n1} & a_{n2} & \cdots & a_{nn} \\
\end{matrix} \right) \]
Az $A$ mátrix determinánsát a következőképpen definiáljuk:
\begin{equation}
\det(A) = \begin{vmatrix}
a_{11} & a_{12} & \cdots & a_{1n} \\
a_{21} & a_{22} & \cdots & a_{2n} \\
\vdots & \vdots & \ddots & \vdots \\
a_{n1} & a_{n2} & \cdots & a_{nn} \\
\end{vmatrix} := \sum_{\sigma \in S_n} (-1)^{\mathcal{I}(\sigma)} \prod_{i = 1}^n a_{i\sigma_i}
\end{equation}
\clearpage
\section{Logikai azonosság}
Tekintsük az $L = \{ 0, 1 \}$ halmazt, és legyenek $a, b, c, d \in L$. Belátjuk a következő azonosságot:\\
\begin{equation}\label{eq:konjunkciok}
(a \wedge b \wedge c) \rightarrow d = a \rightarrow \bigl( b \rightarrow (c \rightarrow d) \bigr) \text{.}
\end{equation}
A következő azonosságokat bizonyítás nélkül használjuk:\\
\begin{subequations}\label{eq:azonossagok}
\begin{equation}\label{eq:implikacio}
x \rightarrow y = \bar{x} \vee y
\end{equation}
\begin{equation}\label{eq:De_Morgan}
\overline{x \vee y} = \bar{x} \wedge \bar{y} \qquad \overline{x \wedge y} = \bar{x} \vee \bar{y}
\end{equation}
\end{subequations}
A (\ref{eq:konjunkciok}) bal oldala, (\ref{eq:azonossagok}) felhasználásával
\begin{equation}\label{eq:levezetes}
(a \wedge b \wedge c) \rightarrow d \underset{(\ref{eq:implikacio})}{=} \overline{a \wedge b \wedge c} \vee d \underset{(\ref{eq:De_Morgan})}{=} (\bar{\mathstrut a} \vee \bar{\mathstrut b} \vee \bar{\mathstrut c}) \vee d \text{.}
\end{equation}
A (\ref{eq:konjunkciok}) jobb oldala, (\ref{eq:implikacio}) ismételt felhasználásával
\begin{align}\label{eq:diszjunkciok}
\begin{split}
a \rightarrow \bigl( b \rightarrow (c \rightarrow d) \bigr) &= \bar{\mathstrut a} \vee \bigl( b \rightarrow (c \rightarrow d) \bigr) \\
&= \bar{\mathstrut a} \vee \bigl( \bar{\mathstrut b} \vee (c \rightarrow d) \bigr) \\
&= \bar{\mathstrut a} \vee \bigl( \bar{\mathstrut b} \vee (\bar{\mathstrut c} \vee d) \bigr) \text{,}
\end{split}
\end{align}
ami a $\vee$ asszociativitása miatt egyenlő (\ref{eq:levezetes}) egyenlettel.
\clearpage
\section{Binomiális tétel}
\begin{subequations}
\begin{align}
(a + b)^{n + 1} &= (a + b) \cdot \left( \sum_{k = 0}^n \binom{n}{k} a^{n - k} b^k \right) \\
&= \cdots \nonumber \\
&= \sum_{k = 0}^n \binom{n}{k} a^{(n + 1) - k} b^k + \sum_{k = 1}^{n + 1} \binom{n}{k - 1} a^{(n + 1) - k} b^{k} \\
&= \cdots \nonumber \\
\begin{split}
&= \binom{n + 1}{0} a^{n + 1 - 0} b^0 + \sum_{k = 1}^n \binom{n + 1}{k} a^{(n + 1) - k} b^k \\
&+ \binom{n + 1}{n + 1} a^{n + 1 - (n + 1)} b^{n + 1} \text{.}
\end{split} \\
&= \sum_{k = 0}^{n + 1} \binom{n + 1}{k} a^{(n + 1) - k} b^k \text{.}
\end{align}
\end{subequations}
\end{document}